\documentclass[11pt,titlepage]{article}
\usepackage[utf8]{inputenc}
\usepackage[dutch]{babel}
\usepackage{amsmath}
\usepackage{amsfonts}
\usepackage{amssymb}
\usepackage{graphicx}
\usepackage[table,xcdraw]{xcolor}
\usepackage[toc,page]{appendix}
\usepackage{hyperref}
\usepackage{listings}
\usepackage{float}
\usepackage{tikz}
\usetikzlibrary{trees}
\usepackage{tikz-qtree}
\usepackage{graphicx}
\usepackage{fancyref}
\usepackage{wrapfig}
\usepackage{url}
\usepackage{pdflscape}
\usepackage{fancyvrb}
\graphicspath{ {Afbeeldingen/} }
\usepackage{subfig}
\usepackage{tabularx}

\newcolumntype{L}[1]{>{\raggedright\arraybackslash}p{#1}}

%% Sets page size and margins 
\usepackage[a4paper,top=3cm,bottom=3cm,left=3cm,right=3cm,marginparwidth=1.75cm]{geometry}

\author{René van Eendenburg 561378 \cr Derk Wiegerinck 567665}

\title{Demo Documentatie}
\usepackage{titling}

\newcommand{\subtitle}[3]{%
	\posttitle{%
		\par\end{center}
	\begin{center}\large#1\end{center}
	\begin{center}\large#2\end{center}
	\begin{center}\large#3\end{center}
	\vskip0.5em}%
}

\subtitle{HAN Arnhem}{Versie 1}{WOR-World}

\frenchspacing
\sloppy
\begin{document}
\maketitle



\tableofcontents
\clearpage


\section{Inleiding}
Dit document beschrijft wat er nodig is om een simulatie te kunnen bouwen, het starten van de simulatie en het uitvoeren van een demo.

\section{Benodigdheden}
Uw Linux distributie zal Robot Operating System (Ros) moeten ondersteunen. Kijk onder deze \href{https://www.ros.org/reps/rep-0003.html}{link} of uw distributie wordt ondersteund.

Verder zijn de volgende onderdelen benodigd om de simulatie te bouwen en uit te voeren
\begin{itemize}
    \item \href{http://wiki.ros.org/melodic/Installation}{Ros Melodic}
    \item GCC 7.4.0 of hoger
\end{itemize}

\section{Bouwen van de broncode}
Allereerst zult u een workspace moeten hebben. Dit kunt u doen door de volgende commando' s uit te voeren

\begin{verbatim}
$ mkdir -p ~/catkin_ws/src
$ cd ~/catkin_ws/
$ catkin_make
$ source /opt/ros/<distro>/setup.bash
\end{verbatim}

Daarna kunt u de packages in de zojuist aangemaakte src map kopieren. 

\begin{verbatim}
$ cp /directory/to/RobotSimulation/src/* ~/catkin_ws/src
$ catkin_make
$ source devel/setup.bash
\end{verbatim}

\section{Uitvoeren simulatie}
Na het bouwen van de broncode, kan de simulatie worden gestart. Hiervoor opent u een terminal in de workspace en voert u de volgende stappen uit.

\begin{verbatim}
$ cd ~/catkin_ws
$ source devel/setup.bash
$ roslaunch al5d_simulation al5d_simulation.launch 
\end{verbatim}

Dit opent Rviz en zal u de robotarm weergeven. Om vervolgens het Demo script te draaien, opent u nog een terminal.

\begin{verbatim}
$ cd ~/catkin_ws
$ source devel/setup.bash
$ cd ~/catkin_ws/src/al5d_simulation/scripts
$ ./demo.bash 
\end{verbatim}


\end{document}
